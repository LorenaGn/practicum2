% CONTEXTUALIZACIÓN
\chapter{Descripción y valoración de la situación educativa}
\section{El centro y el grupo-clase}
El grupo pertenece al segundo curso del GM en Instalaciones Eléctricas y Automáticas, con una asistencia media de 12-15 alumnos. Se observa una diversidad en el nivel de conocimientos y motivación.

\section{Interacciones en el aula}
\textbf{Caso 1: Alumno con problemas de sueño}  
Durante una práctica, observé que un alumno estaba distraído, manipulando un tornillo sin interactuar con la actividad. Al preguntarle, mencionó que no podía pensar ni dormir. Le sugerí acudir a un especialista si tenía dificultades de descanso. Tras la conversación, retomó la actividad.

\textbf{Caso 2: Explicación insuficiente en una práctica}  
Un alumno presentó una respuesta muy escueta en la descripción de un automatismo. Comenté que, desde el punto de vista del cliente, no quedaba claro su funcionamiento. Aunque inicialmente protestó, modificó su respuesta.

\textbf{Caso 3: Reflexión sobre la asignación de prácticas}  
Un alumno me preguntó sobre un problema con disyuntores. Le expliqué que no podía ayudarle y que preguntara a los profesores. Luego, me preguntó por qué, siendo informática, estaba en un ciclo de electricidad. Esto reflejó la falta de coherencia en la asignación de prácticas.
