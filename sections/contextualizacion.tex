% CONTEXTUALIZACIÓN
\chapter{Descripción y valoración de la situación educativa concreta}

Durante el período de prácticas del Prácticum II, mi actividad principal se ha desarrollado en el \textit{CIPFP Canastell}, un centro integrado de Formación Profesional ubicado en San Vicente del Raspeig. En este contexto, he participado en tres módulos de formación pertenecientes a dos ciclos distintos del ámbito de la electricidad y la mecatrónica (Instalación y Mantenimiento):

\begin{itemize}
    \item \textbf{Ciclo de Grado Medio de Instalaciones Eléctricas y Automáticas}:
    \begin{itemize}
      \item Módulo ``Automatismos Industriales'' (1º Curso)
      \item Módulo ``Instalaciones Domóticas'' (2º Curso)
    \end{itemize}
    \item \textbf{Ciclo de Grado Superior de Mecatrónica Industrial}:
    \begin{itemize}
      \item Módulo ``Sistemas Eléctricos y Electrónicos'' (1º Curso)
    \end{itemize}
  \end{itemize}\

La participación se centró principalmente en el módulo de \textit{Instalaciones Domóticas} de segundo curso del Ciclo Formativo de Grado Medio de Instalaciones Eléctricas y Automáticas, y de forma observacional en los otros dos módulos.


\section{Características del alumnado}

Durante el período de prácticas, correspondiente al segundo trimestre del curso académico 2024-25, he podido observar una diferencia significativa entre el número de estudiantes matriculados oficialmente y el número real de asistentes a clase en los diferentes módulos. Según me explicó el tutor, esta situación es habitual en los ciclos formativos, especialmente en primero del Grado Medio, donde el primer trimestre actúa como una especie de filtro natural. Durante ese periodo inicial, suelen concentrarse los alumnos más conflictivos o con menor compromiso académico, lo que convierte las clases en un entorno más complejo para la docencia. Sin embargo, a partir del segundo trimestre, muchos de esos estudiantes se van descolgando progresivamente, lo que conlleva una mejora notable del clima en el aula y una mayor estabilidad en los grupos. Lo cual no siginifica que no haya alumnado conflictivo, sino que la mayoría de ellos ya no asisten a clase.

Mi experiencia como docente en prácticas, al desarrollarse durante este segundo trimestre, se ha dado en un contexto ya más consolidado, con grupos más reducidos, cohesionados y participativos. Esto ha influido positivamente en el desarrollo de las sesiones, especialmente en aquellas en las que he tenido una intervención directa.

En el módulo de \textit{Instalaciones Domóticas} (2º curso del Ciclo Formativo de Grado Medio de Instalaciones Eléctricas y Automáticas), asistían regularmente alrededor de quince estudiantes, la mayoría, mayores de edad, a partir de los diecisiete años. El grupo mostraba una composición heterogénea, tanto en lo académico como en lo experiencial: algunos alumnos accedieron tras finalizar la ESO, otros por prueba de acceso y varios con experiencia previa en el ámbito técnico o laboral. Aunque el interés general por los contenidos era moderado, existían claras diferencias en cuanto a implicación, motivación y ritmo de aprendizaje.

En el módulo de \textit{Automatismos Industriales} (1º curso del mismo ciclo), que seguí desde un enfoque observacional, la asistencia se situaba entre los diecisiete y diecinueve estudiantes, con edades a partir de los dieciséis años. Este grupo se encontraba todavía en proceso de adaptación al entorno y las exigencias de la Formación Profesional, presentando dificultades organizativas, menor autonomía y dispersión en las sesiones prácticas. A pesar de la estabilización tras el primer trimestre, el profesorado destacaba la necesidad de un acompañamiento más intensivo. En este grupo, además, se identificaban casos de alumnado con necesidades educativas específicas, como el de una alumna procedente de la ESO que evidenciaba tanto dificultades académicas como personales, y que requería una atención empática, aunque no habían estrategias de inclusión por parte del equipo docente de manera formal, ya que las adaptaciones en Formación Profesional son de acceso. Sin embargo, sí que había predisposición del tutor y el otro profesor con el que hacía desdoble, de implicarse y realizar apoyo emocional a esta alumna.

Finalmente, en el módulo de \textit{Sistemas Eléctricos y Electrónicos} (1º curso del Ciclo Formativo de Grado Superior de Mecatrónica Industrial), también desde un rol de observadora, la asistencia era más reducida, con unos doce estudiantes a partir de los dieciocho años. Al tratarse de un ciclo superior, el perfil del alumnado era más técnico y profesionalizado. Sin embargo, no estaban exentos de dificultades, especialmente aquellos que compatibilizaban los estudios con un empleo. Esta doble carga provocaba tensiones a la hora de programar exámenes u otras actividades, generando la necesidad de una mayor flexibilidad institucional para responder a las circunstancias personales del alumnado.

En conjunto, el alumnado con el que tuve contacto reflejaba la diversidad y complejidad propias de los ciclos de Formación Profesional, tanto en lo académico como en lo personal, y me permitió observar y reflexionar sobre diferentes estrategias de gestión del aula, acompañamiento individualizado e inclusión educativa.

A continuación se presentan algunos casos concretos que ilustran la diversidad del alumnado y las dinámicas observadas en el aula:

\textit{Caso 1: Alumno con problemas de sueño}.  
Durante una práctica, observé que un alumno estaba distraído, manipulando un tornillo sin interactuar con la actividad. Al preguntarle, mencionó que no podía pensar ni dormir. Le sugerí acudir a un especialista si tenía dificultades de descanso. Tras la conversación, retomó la actividad.

\textit{Caso 2: Explicación insuficiente en una práctica}. 
Un alumno presentó una respuesta muy escueta en la descripción de un automatismo. Comenté que, desde el punto de vista del cliente, no quedaba claro su funcionamiento. Aunque inicialmente protestó, modificó su respuesta.

\textit{Caso 3: Reflexión sobre la asignación de prácticas}.  
Un alumno me preguntó sobre un problema con disyuntores. Le expliqué que no podía ayudarle y que preguntara a los profesores. Luego, me preguntó por qué, siendo informática, estaba en un ciclo de electricidad. Esto reflejó la falta de coherencia en la asignación de prácticas.



\section{Condiciones físicas del aula}

Las sesiones en las que he participado, tanto en calidad de observadora como en mi intervención activa, se han desarrollado en diferentes espacios del CIPFP Canastell, adaptados a las necesidades técnicas y metodológicas de cada módulo. Todos ellos se corresponden con aulas-taller, con una distribución funcional del espacio que facilita el aprendizaje práctico, aunque también presentan ciertas limitaciones derivadas de la antigüedad de las instalaciones.

El módulo de \textit{Instalaciones Domóticas} se imparte en un aula-taller especializada, en cuyo almacén se conservan diversos materiales destinados a las prácticas del alumnado, como componentes domóticos (enchufes inteligentes, bombillas, routers, etc.) y otros elementos como dispositivos KNX. Para el desarrollo de la situación de aprendizaje centrada en la ciberseguridad en redes domóticas, configuré un router independiente, sin conexión a Internet, que generaba un SSID denominado \textit{RedDomotica}. Este entorno simulado permitió al alumnado experimentar con la conexión y configuración de dispositivos inteligentes en condiciones seguras. La disposición del aula favorece el trabajo individual y por parejas, y se complementa con el uso de una pizarra digital de última generación, modelo \textit{SYNETECH Advance Aquarius} (A7532), incorporada este mismo curso gracias a los fondos europeos destinados a la mejora de la Formación Profesional. Esta herramienta ha supuesto una mejora significativa en la dinámica de clase, ya que permite, entre otras funcionalidades, compartir las anotaciones de la pizarra mediante un código QR, lo que facilita que el alumnado pueda centrarse en la explicación sin preocuparse por copiar apuntes en tiempo real.

En el módulo de \textit{Automatismos Industriales}, las prácticas se llevan a cabo en un taller con mesas técnicas equipadas con cuadros eléctricos, pulsadores, relés y autómatas programables. Durante una de las sesiones, detectamos que algunos tableros estaban sueltos debido a la falta de tornillos, por lo que mi tutor y yo procedimos a su reparación. Mientras tanto, el alumnado continuó con las prácticas que tenían pendientes. Este tipo de incidencias materiales es relativamente frecuente en el centro, y se entienden dentro del contexto de unas instalaciones con más de 40 años de antigüedad. De hecho, algunos elementos del mobiliario, como taburetes o mesas, conservan aún el etiquetado original de fabricación de los años 80, lo que ilustra bien la necesidad de una renovación progresiva de infraestructuras.

En el módulo de \textit{Sistemas Eléctricos y Electrónicos}, las clases se desarrollaban en un aula-taller adecuada en cuanto a espacio, aunque sin demasiadas novedades en cuanto a equipamiento. El número reducido de estudiantes favorecía un ambiente tranquilo y propicio para la formación, aunque las sesiones de teoría prolongada dificultaban en ocasiones el mantenimiento de la atención. En este caso, la estructura tradicional del aula contrastaba con las mejoras tecnológicas observadas en otros espacios del centro.

En conjunto, puede afirmarse que los espacios utilizados están ajustados a las necesidades de los distintos módulos, si bien el desgaste acumulado con el paso de los años y el uso intensivo hacen necesaria una inversión sostenida para mantener su funcionalidad. Este objetivo forma parte de las prioridades estratégicas recogidas tanto en el \textit{Manual de Calidad} \cite{politicaCalidadCanastell} como en el \textit{Proyecto de Dirección} \cite{proyectoDireccion2022} del centro.
