\chapter{El departamento didáctico y el proceso de enseñanza-aprendizaje}

\section{El departamento didáctico}
El departamento de Electricidad-Electrónica del CIPFP Canastell, al que están adscritos los módulos en los que participé durante el Prácticum, destaca por su cohesión y compañerismo. El ambiente general entre el profesorado es cordial, lo que favorece la colaboración y el intercambio de ideas, incluso con profesorado recién incorporado o con perfiles más técnicos que docentes.

A nivel organizativo, el departamento dispone de documentación clara y actualizada, como las programaciones didácticas, criterios de evaluación comunes, rúbricas y listados de prácticas subidos a la plataforma AULES. La planificación incluye también aspectos como la recuperación de resultados de aprendizaje o la adaptación a necesidades individuales, en línea con los principios del modelo de calidad implantado en el centro.

Desde el Proyecto de Dirección \cite{proyectoDireccion2022} se promueve una línea de trabajo centrada en el análisis y detección de necesidades formativas del profesorado para implementar metodologías activas y avanzar en la personalización del aprendizaje. Sin embargo, y aunque existen experiencias concretas como el aprendizaje basado en retos en el módulo de Robótica, esta implementación se encuentra aún en una fase incipiente. La propia docente de dicho módulo reconocía no haber recibido formación específica y aplicar este enfoque “como puede”.

A pesar de los esfuerzos del centro por avanzar en esa dirección, la metodología dominante sigue siendo la clase magistral, especialmente en los módulos más técnicos. Las prácticas se desarrollan a partir de listados que los alumnos autogestionan, y el profesorado se dedica a resolver dudas y evaluar los avances. Esta dinámica puede encontrarse incluso en módulos donde se combina teoría y resolución de problemas, como Domótica o Sistemas Eléctricos.


\section{Proceso de enseñanza-aprendizaje}

Desde el punto de vista metodológico, las clases observadas muestran un enfoque predominantemente práctico, con una fuerte orientación al desarrollo de competencias técnicas y profesionales. Las sesiones suelen alternar teoría aplicada con la realización de prácticas de taller, siendo este último el eje central del proceso de enseñanza-aprendizaje en los ciclos formativos.

En el módulo de \textit{Instalaciones Domóticas}, se trabaja mucho con simulaciones y dispositivos reales, lo que motiva especialmente al alumnado. En este contexto, tuve la oportunidad de aplicar una situación de aprendizaje centrada en la ciberseguridad de redes domóticas, lo que me permitió integrar herramientas digitales, metodologías activas y dinámicas participativas. La respuesta del alumnado fue positiva, y se generó un ambiente de curiosidad y colaboración.

En los módulos observados, como \textit{Automatismos Industriales} o \textit{Sistemas Eléctricos y Electrónicos}, se aprecian estilos docentes variados, aunque predominan las explicaciones estructuradas apoyadas en recursos gráficos, junto con la resolución de prácticas. Las relaciones profesorado-alumnado son cercanas, con una comunicación fluida que permite atender tanto dudas técnicas como cuestiones personales, cuando estas interfieren en el proceso de aprendizaje.

Se detectan, no obstante, algunas dificultades relacionadas con la motivación de parte del alumnado, especialmente en los primeros cursos del Grado Medio. El acompañamiento constante y la insistencia en la adquisición de hábitos de trabajo son estrategias que el profesorado aplica con frecuencia. Por otro lado, se observa una buena disposición para atender a estudiantes con necesidades educativas específicas, mostrando sensibilidad ante sus circunstancias.

En general, la enseñanza en el centro se sustenta en la idea de aprender haciendo, y pone el foco en la adquisición de competencias útiles para el entorno profesional real, en consonancia con los principios de la Formación Profesional Dual, aunque no todos los grupos participen directamente de esta modalidad.
