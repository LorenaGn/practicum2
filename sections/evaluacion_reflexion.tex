% ANÁLISIS Y VALORACIÓN DE LA INTERVENCIÓN
\chapter{Análisis y valoración de la intervención}

\section{Reflexión sobre la intervención como docente en prácticas}

Mi participación durante el Prácticum II ha sido fundamentalmente de observación activa, aunque también he podido intervenir de forma puntual como apoyo a la docencia y, de forma más significativa, en la planificación, desarrollo y evaluación de la unidad de trabajo sobre seguridad en dispositivos domóticos. Esta intervención me ha permitido poner en práctica mis conocimientos técnicos desde una perspectiva didáctica, reflexionar sobre el funcionamiento real de un aula de Formación Profesional y adaptar mi comunicación y estrategias a un perfil de alumnado diverso.

A lo largo del periodo de prácticas, observé dinámicas muy diferentes entre los grupos, especialmente entre primer y segundo curso del ciclo de grado medio. En algunos casos, la desmotivación o el uso inadecuado del móvil dificultaban el ritmo de las clases, mientras que en otros momentos la implicación era notable, sobre todo cuando las actividades se conectaban con su entorno cotidiano o se planteaban en formato de reto.

La experiencia más significativa fue sin duda la unidad de trabajo que diseñé y llevé a cabo en el módulo de \textit{Instalaciones Domóticas}. Su planificación, desarrollo y evaluación me permitieron explorar todo el ciclo didáctico y afrontar imprevistos reales (como problemas técnicos o intervenciones espontáneas del alumnado), lo que enriqueció enormemente mi formación como futura docente.

\section{Valoración de las propuestas pedagógicas del departamento}

El Departamento de Electricidad y Electrónica muestra un compromiso activo con la mejora continua y con la aplicación de metodologías que favorezcan el aprendizaje práctico. A pesar de que el enfoque dominante sigue siendo la clase magistral tradicional, se observan esfuerzos por incorporar metodologías activas, como el aprendizaje basado en retos (ABR), especialmente en módulos como Robótica.

Sin embargo, también se detectan limitaciones estructurales y de formación docente que condicionan la implantación de enfoques más innovadores. Por ejemplo, una profesora comentó que aplicaba ABR sin haber recibido formación específica, lo que puede limitar su potencial. Aun así, se aprecia una buena disposición del profesorado a colaborar, compartir materiales y debatir sobre estrategias didácticas, lo cual es una fortaleza clara del departamento.

\section{Valoración de la programación de aula y la situación de aprendizaje}

La programación del módulo de Instalaciones Domóticas establece un marco coherente con el currículo oficial y recoge una organización de contenidos por unidades de trabajo. Mi unidad de trabajo se diseñó de forma complementaria a la programación y fue validada por el tutor, quien valoró positivamente su enfoque competencial y su conexión con temas de actualidad.

La propuesta encajó perfectamente dentro de los objetivos del módulo, trabajando competencias técnicas, actitudes profesionales y elementos transversales como la seguridad digital. La planificación fue realista en tiempos y recursos, aunque la práctica evidenció la necesidad de prever más profundamente los posibles bloqueos o problemas técnicos derivados de las condiciones del entorno (p.ej., limitaciones del router).

\section{Elaboración y uso de materiales didácticos}

Durante la intervención diseñé y utilicé una presentación propia, con estructura clara, preguntas de reflexión y vídeos breves que facilitaron el debate. También preparé un entorno de red simulado para la parte práctica, lo cual permitió reproducir un contexto de aprendizaje próximo a la realidad profesional.

Los materiales fueron bien acogidos por el alumnado, especialmente por su componente visual y participativo. Además, la nube de palabras inicial y el uso del móvil con fines académicos aportaron un valor añadido a la dinámica.

En futuras intervenciones, sería interesante sistematizar el uso de estos materiales mediante plantillas o rúbricas que permitan replicar y evaluar la experiencia con más precisión.

\section{Valoración del desarrollo de las clases}

Las sesiones se desarrollaron de forma fluida, combinando explicación teórica con actividad práctica. Se fomentó un clima participativo y respetuoso, en el que los estudiantes pudieron expresarse, proponer ideas y colaborar entre sí. Las intervenciones del alumnado fueron en general pertinentes y mostraron interés real por los aspectos técnicos de la seguridad en entornos conectados.

La sesión práctica, en concreto, generó mucho dinamismo y curiosidad. Incluso situaciones no previstas, como el cambio de SSID o el bloqueo del router, se transformaron en oportunidades de aprendizaje valiosas. Esto refuerza la idea de que el aprendizaje en FP debe estar abierto a lo imprevisto, ya que la realidad profesional no siempre sigue un guion fijo.

\section{Evaluación del aprendizaje del alumnado}

Aunque la evaluación formal se limitó a tres preguntas tipo test en el examen del módulo, la actividad permitió observar aprendizajes relevantes no siempre fácilmente evaluables. El alumnado adquirió vocabulario técnico nuevo, comprendió la importancia de la seguridad en sistemas conectados y mostró capacidad de aplicar procedimientos de configuración con criterio.

De cara a futuras intervenciones y diseño de unidades de trabajo, se podría complementar esta evaluación con una rúbrica sencilla o una autoevaluación del alumnado para recoger su percepción del aprendizaje y su implicación.

\section{Consideración sobre criterios DUA}

Durante esta unidad no se aplicaron criterios explícitos del Diseño Universal para el Aprendizaje (DUA). No obstante, sí se favoreció el acceso a la información por múltiples vías (oral, visual, práctica), se permitió el trabajo en grupo y se ofrecieron apoyos individuales durante la actividad. En un futuro, sería interesante incorporar elementos del DUA de forma más sistemática, como permitir distintas formas de expresión o adaptar tareas según niveles de dominio técnico.

\section{Propuestas y sugerencias de mejora}

A partir de esta experiencia, propongo las siguientes mejoras tanto a nivel didáctico como organizativo:

\begin{itemize}
  \item Incluir más contenidos sobre ciberseguridad en el currículo de domótica, no como unidad aislada sino de forma transversal.
  \item Facilitar formación específica al profesorado en metodologías activas como el aprendizaje basado en retos o el uso educativo de tecnologías personales.
  \item Potenciar la reflexión ética y crítica en el uso de dispositivos conectados, incluyendo casos reales, dilemas o debates.
  \item Mejorar la asignación de plazas de prácticas, procurando una correspondencia más ajustada entre el perfil del estudiante y el módulo en el que se le asigna.
  \item Documentar buenas prácticas como esta unidad de trabajo, para que puedan ser replicadas y adaptadas por otros docentes.
\end{itemize}
