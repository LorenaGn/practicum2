% RESUMEN
\chapter{Resumen}

Esta memoria recoge la experiencia y las intervenciones realizadas durante el Prácticum II en el CIPFP Canastell, dentro del módulo de \textit{Instalaciones Domóticas} del segundo curso del Grado Medio en Instalaciones Eléctricas y Automáticas. La intervención se centró en una situación de aprendizaje que abordó los riesgos de seguridad en la domótica de bajo coste, permitiendo al alumnado analizar vulnerabilidades en dispositivos IoT y aplicar buenas prácticas de ciberseguridad.

A lo largo del período de prácticas, mi rol principal fue de observadora, ya que mi formación en Ingeniería Informática no estaba alineada con el contenido eléctrico del ciclo. Sin embargo, esto no impidió que pudiera desarrollar una intervención concreta en un área más afín a mi perfil: la seguridad en domótica de bajo coste. 

La situación de sprendizaje titulada "Domótica de Bajo Coste: ¿Comodidad Inteligente o Riesgo Silencioso?", se llevó a cabo el 6 de febrero de 2025. En esta sesión, el alumnado analizó vulnerabilidades de dispositivos IoT de bajo coste y aplicó buenas prácticas en ciberseguridad. La actividad combinó una parte teórica donde se explicaron los riesgos y ataques más comunes en redes domóticas, y una parte práctica en aula-taller, en la que los estudiantes experimentaron con la configuración y seguridad de un \textit{enchufe inteligente Shelly Plug} y un \textit{router WiFi}, identificando riesgos potenciales y simulando ataques de fuerza bruta.

Se utilizaron estrategias didácticas activas, fomentando la participación del alumnado mediante dinámicas interactivas y herramientas digitales. La evaluación se realizó a través de una prueba tipo test y la observación del desempeño práctico.

Los resultados evidencian que la actividad resultó altamente efectiva en la concienciación sobre la ciberseguridad en el ámbito domótico, con una participación notable en la parte práctica. No obstante, se identificaron dificultades en la motivación de ciertos alumnos y en la gestión del tiempo en la actividad práctica. 

En conclusión, esta experiencia ha permitido desarrollar competencias pedagógicas clave, mejorando la capacidad de planificación, adaptación metodológica y gestión del aula en un entorno de Formación Profesional.

% \textbf{Palabras clave}: Prácticum II, Formación Profesional, domótica, ciberseguridad, enseñanza práctica, aprendizaje basado en problemas.



% Durante la observación en el aula, pude identificar dinámicas interesantes en la interacción profesor-alumno, así como detectar dificultades individuales. Entre los casos destacados:
% \begin{itemize}
%     \item Un alumno de \textit{1º GM} mostraba signos de falta de concentración y comentó que no podía dormir bien. Tras una breve conversación en la que le sugerí buscar ayuda profesional, retomó la práctica con mayor enfoque.
%     \item Otro estudiante de \textit{1º GM} presentó una descripción ambigua en una práctica. Tras hacerle reflexionar sobre la claridad de su explicación desde el punto de vista de un cliente, modificó su respuesta.
%     \item Un alumno me llamó \textit{"profe"} y me preguntó sobre disyuntores. Tras explicarle que no podía ayudarle porque no era mi área, entablamos una conversación en la que cuestionó por qué me habían asignado a un módulo eléctrico, reflejando la falta de alineación entre mi perfil y la asignación de prácticas.
% \end{itemize}

% La evaluación de la actividad evidenció que la parte práctica generó gran interés en el alumnado, destacando la importancia de incluir más contenido de ciberseguridad en la formación de Instalaciones Domóticas. A pesar de las limitaciones derivadas de mi perfil técnico, esta experiencia me permitió reflexionar sobre el papel del docente en FP, el enfoque metodológico en enseñanza práctica y la importancia de la motivación y adaptación de los contenidos a las necesidades del alumnado.

