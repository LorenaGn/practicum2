% RESUMEN
\chapter{Resumen}

Esta memoria recoge la experiencia y las intervenciones realizadas durante el Prácticum II en el CIPFP Canastell. Aunque mi rol principal fue el de observadora, esto no impidió que pudiera desarrollar una intervención concreta en un área más afín a mi perfil. La intervención se centró en una situación de aprendizaje que abordó los riesgos de seguridad en la domótica de bajo coste, permitiendo al alumnado analizar vulnerabilidades en dispositivos IoT y aplicar buenas prácticas de ciberseguridad. 

Se utilizaron estrategias didácticas activas, fomentando la participación del alumnado mediante dinámicas interactivas y herramientas digitales. La evaluación se realizó a través de una prueba tipo test y la observación del desempeño práctico.

Los resultados evidencian que la actividad resultó altamente efectiva en la concienciación sobre la ciberseguridad en el ámbito domótico, con una participación notable en la parte práctica. No obstante, se identificaron dificultades en la motivación de ciertos alumnos y en la gestión del tiempo en la actividad práctica. 

Además de la propuesta didáctica realizada, se incluyen reflexiones sobre la observación de clases y la interacción con el alumnado, características de la situación educativa, como cursos, etapas número de alumnado junto con sus características, las singularidades de los espacios y recursos como el aula, el taller, el departamento, entre otros.

En conclusión, esta experiencia ha permitido desarrollar competencias pedagógicas clave, mejorando la capacidad de planificación, adaptación metodológica y gestión del aula en un entorno de Formación Profesional.
