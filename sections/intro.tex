% INTRODUCCIÓN
\chapter{Introducción}
El presente documento recoge la experiencia desarrollada durante el período del Prácticum II correspondiente al \textit{Máster en Profesorado de Educación Secundaria Obligatoria y Bachillerato, Formación Profesional y Enseñanza de Idiomas}, llevado a cabo en el CIPFP Canastell, centro integrado de Formación Profesional ubicado en San Vicente del Raspeig. Este centro destaca por su amplia oferta formativa adaptada al contexto industrial y empresarial local, con especial énfasis en las familias profesionales relacionadas con la electricidad y electrónica, entre otras.

Durante mi estancia en el centro, mi rol principal fue el de observadora, participando de manera puntual en algunas intervenciones pedagógicas. Este enfoque fue consecuencia directa de la falta de adecuación entre mi perfil formativo, Ingeniería Informática, y el contenido curricular mayoritariamente eléctrico de los ciclos en los que se desarrolló la práctica docente. Esta situación no solo condicionó mi nivel de participación directa en la docencia, sino que también generó una reflexión crítica sobre la importancia de alinear los perfiles de formación del profesorado en prácticas con las asignaturas asignadas.

Mi intervención principal tuvo lugar en el módulo de \textit{Instalaciones Domóticas}, impartido en el segundo curso del Grado Medio en Instalaciones Eléctricas y Automáticas, durante una sesión realizada el 6 de febrero de 2025. Esta actividad, titulada "Domótica de Bajo Coste: ¿Comodidad Inteligente o Riesgo Silencioso?", permitió al alumnado explorar aspectos fundamentales de ciberseguridad aplicados a la domótica de bajo coste, tema más próximo a mi ámbito de formación.

En dicha situación de aprendizaje se desarrollaron diversas dinámicas que facilitaron la interacción y participación activa del alumnado. Entre estas destacó la utilización de herramientas digitales como un código QR que les permitió contribuir a una nube de palabras sobre riesgos de seguridad en domótica. Este tipo de recursos sirvió como rompehielo para generar un ambiente propicio para el aprendizaje activo. Además, la práctica consistió en configurar un enchufe inteligente \textit{Shelly Plug}, conectarlo a una red WiFi con seguridad débil y simular ataques informáticos básicos, actividades que generaron un gran interés en las/os estudiantes, evidenciado en la alta participación y la profundidad de las reflexiones posteriores.
Cabe mencionar que hubo un estudiante que fue capaz de cambiarle el nombre a la red, haciendo que no nos pudiéramos conectar al enchufe inteligente. Al finalizar la clase, se acercó para preguntarme si había "molestado" con su acción, a lo que le respondí que no, que había sido una acción muy interesante y que era una forma de aprender.

Por otro lado, la observación directa de dinámicas de enseñanza en el aula de 1º GM, específicamente en el módulo de \textit{Automatismos Industriales}, aportó experiencias valiosas sobre la realidad cotidiana del profesorado en FP. Durante estas sesiones observé situaciones que enriquecieron mi visión acerca de la complejidad de la labor docente, especialmente cuando se trata de estudiantes con diferentes necesidades y motivaciones. Entre estas situaciones destaco las siguientes anécdotas registradas en mi cuaderno de bitácora:

\begin{itemize} %[leftmargin=*, label=\textbullet]
    \item En una sesión práctica, observé a un alumno de 1º GM aparentemente desconectado de la actividad. Al acercarme, me comentó que tenía problemas de sueño y dificultades para concentrarse. Aproveché la oportunidad para recomendarle que buscara ayuda profesional, sugiriéndole acudir al médico o a un especialista. Tras esta breve conversación y sin mediar palabras, noté un cambio positivo en su actitud, retomando la actividad con mayor atención y disposición.
    
    \item Otra situación se produjo cuando un alumno presentó una descripción escueta y ambigua en la explicación de un automatismo. Al pedirme el docente que la revisara, le hice ver al estudiante que su descripción no era suficiente desde la perspectiva de un cliente, destacando la importancia de una buena comunicación técnica. Aunque inicialmente hubo cierta resistencia, finalmente aceptó mejorar su respuesta gracias a la insistencia del profesor titular.
    
    \item También recuerdo con especial ilusión cuando uno de los alumnos se refirió a mí como ''profe'' para preguntarme sobre una cuestión técnica relacionada con disyuntores. Dado que era un tema fuera de mi área de especialización, le aclaré que no podía ayudarle y le remití al profesorado titular. Esta situación derivó en una conversación en la que el alumno, con notable lógica y observación, cuestionó por qué me habían asignado un módulo de electricidad siendo ingeniera informática, mostrando la clara falta de adecuación en la asignación de mis prácticas.
    
    \item En otras sesiones, observé situaciones diversas, como alumnxs que mostraban estrés por exámenes o tareas, y pude apreciar cómo el profesorado abordaba estas situaciones desde una perspectiva no solo académica, sino también emocional, enfatizando la importancia de un acompañamiento integral a lxs estudiantes en FP.
\end{itemize}

Estas anécdotas reflejan no solo aspectos concretos de la realidad cotidiana del aula, sino también la importancia de la figura docente en Formación Profesional como agente clave en el desarrollo integral del alumnado, desde lo académico hasta lo personal y emocional.

Finalmente, cabe destacar que esta experiencia ha supuesto una oportunidad única para reflexionar sobre las dinámicas didácticas empleadas en FP, la necesidad de adaptar los contenidos a las características y motivaciones del alumnado, así como para entender la trascendencia de ofrecer un contexto de aprendizaje seguro y propicio para la participación. A pesar de las limitaciones derivadas de mi formación inicial, esta experiencia ha sido enriquecedora en múltiples dimensiones, contribuyendo notablemente a mi formación pedagógica y proporcionando aprendizajes que serán aplicables en futuros contextos docentes.
