% Autoevaluación y conclusiones
\chapter{Autoevaluación y conclusiones}


% Autoevaluación
\section{Autoevaluación}

Durante el desarrollo del Prácticum II he tenido la oportunidad de profundizar en el funcionamiento real de la Formación Profesional, tanto a nivel organizativo como metodológico. Ha sido una experiencia intensa y transformadora que ha exigido de mí flexibilidad, capacidad de adaptación y voluntad constante de aprendizaje. La mayor dificultad encontrada fue la desconexión inicial entre mi perfil profesional —procedente del ámbito de la Ingeniería Informática— y los contenidos técnicos impartidos en el módulo, centrados en instalaciones eléctricas. No obstante, esta aparente limitación se convirtió en una oportunidad para aportar una visión complementaria, especialmente en aspectos vinculados a la ciberseguridad y al uso responsable de dispositivos IoT.

Aunque mi papel fue principalmente de observadora en muchos momentos, intenté implicarme en todo lo posible. En las sesiones prácticas, me paseaba por el aula para ver cómo trabajaban los estudiantes, observar las estrategias del profesorado y resolver pequeñas dudas cuando era posible. En las sesiones más teóricas, aproveché para aprender sobre los contenidos del ciclo —completamente nuevos para mí— o para investigar aspectos relacionados con la estructura del centro, la documentación institucional y el funcionamiento en general.

Una anécdota significativa que me marcó especialmente fue cuando uno de los alumnos, de forma espontánea, se dirigió a mí como “profe”. Ese gesto, aunque pequeño, supuso un reconocimiento implícito a mi presencia y mi implicación en el aula. También me resultó gratificante haber podido acompañar, de forma muy genérica, a un alumno y a una alumna en situaciones personales en las que simplemente escuchar o mostrar interés fue valorado por ellos.

Cabe señalar que la preparación de la práctica requirió un trabajo previo importante. Inicialmente adquirí los materiales por mi cuenta, incluyendo un enchufe inteligente y un router antiguo proporcionado por una operadora. Dediqué varias horas en casa a la configuración de la red y los dispositivos, tratando de crear un entorno seguro, cerrado y funcional. Uno de los mayores retos fue descubrir que muchos de los productos comercializados en grandes superficies como Leroy Merlin u Obramat dependen completamente de sus aplicaciones propias, las cuales requieren conexión a Internet y transmiten datos a la nube. Esto hizo inviable su uso para una práctica local sin acceso externo, y me obligó a buscar alternativas más versátiles como el enchufe Shelly Plug, que finalmente utilicé con éxito.

Por último, señalar que las dos fases de prácticas (Prácticum I y II) han estado en mi caso unificadas en una sola experiencia continua, lo cual ha permitido una mayor inmersión en el entorno educativo y una consolidación progresiva de mi rol como futura docente.


% Conclusiones
\section{Conclusiones}

La experiencia ha sido profundamente enriquecedora, tanto desde el punto de vista técnico como humano. El diseño, implementación y evaluación de la unidad de trabajo me ha permitido desarrollar competencias pedagógicas clave, desde la planificación didáctica hasta la gestión del aula, pasando por la elaboración de materiales propios y la evaluación del alumnado.

La unidad ha contribuido a conectar los contenidos curriculares con la realidad tecnológica del presente, fomentando el pensamiento crítico, la reflexión ética y la conciencia sobre la seguridad digital. El alto grado de implicación del alumnado, junto con la aparición de imprevistos técnicos que fueron aprovechados como oportunidades didácticas, ha reforzado la idea de que el aprendizaje significativo no siempre sigue un guion previsible.

Considero especialmente relevante la importancia de ajustar mejor las asignaciones de prácticas al perfil del estudiante, así como de seguir promoviendo metodologías activas, evaluaciones auténticas y una visión integradora de la tecnología en la educación técnica.

Este Prácticum II ha supuesto para mí un punto de inflexión, consolidando mi decisión de orientar mi futuro profesional hacia la docencia y reafirmando mi compromiso con una educación técnica de calidad, humana y con conciencia crítica.

