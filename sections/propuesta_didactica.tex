% SITUACIÓN DE APRENDIZAJE
\chapter{Propuesta didáctica: Domótica de Bajo Coste}
\section{Justificación y Contexto}
El uso de dispositivos domóticos de bajo coste está en auge, pero la seguridad sigue siendo un aspecto desatendido. Esta SA busca concienciar a los alumnos sobre las vulnerabilidades de estos sistemas.

\section{Objetivos}
\begin{itemize}
    \item Comprender los riesgos de seguridad en la domótica de bajo coste.
    \item Identificar vulnerabilidades en dispositivos IoT.
    \item Aplicar buenas prácticas de configuración segura.
\end{itemize}

\section{Contenidos}
\begin{itemize}
    \item Seguridad en redes WiFi y dispositivos inteligentes.
    \item Métodos de ataque: fuerza bruta, sniffing, spoofing.
    \item Configuración de dispositivos domóticos seguros.
\end{itemize}

\section{Desarrollo de la sesión}
\textbf{1. Teoría (1h)}  
Explicación sobre vulnerabilidades en domótica y medidas de protección.

\textbf{2. Práctica (1h 45min)}  
\begin{itemize}
    \item Configuración de un enchufe inteligente Shelly Plug.
    \item Simulación de ataques a una red WiFi insegura.
    \item Análisis de permisos de aplicaciones domóticas.
\end{itemize}

\section{Evaluación}
\begin{itemize}
    \item Participación en la sesión (30\%).
    \item Realización de la práctica (40\%).
    \item Examen tipo test (30\%).
\end{itemize}