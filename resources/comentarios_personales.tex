
% Cositas que podría añadir:
%   - Libro te texto: automatismos industriales juan carlos martin 
%     (https://www.academia.edu/25701944/Automatismos_industriales)
%     (https://es.slideshare.net/slideshow/automatismos-industriales-juan-martin-y-maria-garcia-pdf/270378889)
%   - ORDEN de 29 de julio 2009, de la Conselleria de Educación por la que se establece para la Comunitat Valenciana
%   el currículo del ciclo formativo de Grado Medio correspondiente al título de Técnico en Instalaciones Eléctricas
%   y Automáticas. [2009/9814] -- https://dogv.gva.es/datos/2009/09/02/pdf/2009_9814.pdf
%   - REAL DECRETO 177/2008, de 8 de febrero, por
%   el que se establece el título de Técnico en Instalaciones Eléctricas y Automáticas y se fijan sus
%   enseñanzas mínimas - https://www.boe.es/eli/es/rd/2008/02/08/177/dof/spa/pdf
%   - ORDEN 12/2022, de 9 de marzo, de la Conselleria de
%   Educación, Cultura y Deporte, por la que se regula el
%   módulo profesional de Formación en Centros de Trabajo
%   (FCT) de los ciclos formativos de grado medio y superior,
%   Formación Profesional Básica, Programas Formativos de
%   Cualificación Básica, Cursos de Especialización y Bloque de Formación Práctica (BFP) de las Enseñanzas de
%   Régimen Especial, en el ámbito territorial de la Comunitat
%   Valenciana. [2022/2086] -- https://dogv.gva.es/datos/2022/03/16/pdf/2022_2086.pdf




% % Intervención

%         \textbf{Actividades}:
%         \begin{itemize}
%             \item \textbf{Actividad 1}: Presentación de conceptos, características de los dispositivos domóticos de bajo coste, reflexión sobre vulnerabiidades y consejos básicos para reducir riesgos. Será un trabajo en grupo-clase donde todo el alumnado participará en las reflexiones y cuestiones planteadas. La presentación será proyectada en la pizarra electrónica, teniendo varios puntos de interacción con el alumando para mantener el interés y la atención: al inicio de la presentación, y a modo de rompehielo, se utilizará un código QR a través del cual los alumnos podrán participar en una nube de palabras a través de su dispositvo móvil; en otro momento, se lanzarán interrogantes sobre cuestiones relacionadas con los riesgos de seguridad y saldrán a la pizarra para apuntar lo que se les ha ocurrido.

%             \item \textbf{Actividad 2}: Actividad Práctica de una instalación domótica básica. Con unos pocos dispositivos (como un router, y una bombilla inteligente o similar) se intentará configurar una red simulando una red insegura donde los alumnos podrán "hackear" la red y aprender por qué es importante tener en consideración los aspectos de seguridad. Intentarás conectar la bombilla inteligente a la red, pasando por la instalación de la app correspondiente que les solicitará permisos de WiFi, Bluetooth, Localización, etc. Se realizará con un móvil que no sea del propio alumnado para no comprometer su provacidad. De esta forma, podrán analizar y reflexionar sobre por qué pueden estas aplicaciones solicitar dichos datos, si son seguros, si respetan la privacidad, etc.
%             \begin{itemize}
%                 \item \textbf{Prueba 1}: router huawei (vodafone 2012) no conectado a Internet, configurado con un SSID denominado "RedDomotica" con contraseña débil (12345678). Bombilla inteligente y enchufe inteligente obramart. 
%                 \begin{itemize}
%                     \item Instalación de la app Tuya Smart (se requiere registro)
%                     \item Instalación de la app NavigationSmartHome (para la bombilla) (se requiere registro)
%                     \item La instalación de las app requiere acceso a WiFi, Bluetooth y Localización.
%                     \item El móvil se conecta a la red RedDomotica, la cual no tiene acceso a Internet.
%                     \item Se enciende el dispositivo, primero la bombilla y luego el enchufe por separado. Con el móvil conectado a la red 
%                 \end{itemize}

%                 \item \textbf{Prueba 2}: router huawei (vodafone 2012) configurado como switch, con el mismo SSID y mismo password, ahora con acceso a Internet. Bombilla inteligente y enchufe inteligente obramart. 
%                 \begin{itemize}
%                     \item Instlada la app Tuya Smart
%                     \item El móvil se conecta a la red RedDomotica, la cual no tiene acceso a Internet.
%                     \item Misma operación que en el caso anterior. Parece ser que el indicador de progreso avanza algo más, pero falla igualmente el emparejamiento.
%                 \end{itemize} 

%                 \item \textbf{Prueba 3}:  router tp-link. Configurado con SSID "RedDomotica" y contraseña débil, sin acceso a Internet. Enchufe inteligente Selly Plug y app de la misma marca. 
%                 \begin{itemize}
%                     \item Instalación de la app Shelly (requiere registro: creación de la cuenta clasedomotica6@gmail.com - Qwerty12345!)
%                     \item Se conecta el enchufe inteligente a la corriente. 
%                     \item Se conecta el móvil a RedDomotica
%                     \item Se intenta asignar la red RedSomotica al enchufe.
%                 \end{itemize} 
%             \end{itemize}

%         \end{itemize}
%         \\
%         \hline



% % Autoevaluación y conclusiones
% \chapter{Autoevaluación y Conclusiones}

% % Bibliografía
% \chapter*{Bibliografía}
% \begin{itemize}
%   \item CIPFP Canastell. "Organización del centro". Recuperado de: \url{https://portal.edu.gva.es/cipfpcanastell/informacion-general/organizacion-del-centro/}
%   \item Proyecto de Dirección CIPFP Canastell 2022-2026. Documento interno del centro.
% \end{itemize}



% \end{document}



\chapter{Resumen}
% La situación de sprendizaje titulada "Domótica de Bajo Coste: ¿Comodidad Inteligente o Riesgo Silencioso?", se llevó a cabo el 6 de febrero de 2025. En esta sesión, el alumnado analizó vulnerabilidades de dispositivos IoT de bajo coste y aplicó buenas prácticas en ciberseguridad. La actividad combinó una parte teórica donde se explicaron los riesgos y ataques más comunes en redes domóticas, y una parte práctica en aula-taller, en la que los estudiantes experimentaron con la configuración y seguridad de un \textit{enchufe inteligente Shelly Plug} y un \textit{router WiFi}, identificando riesgos potenciales y simulando ataques de fuerza bruta.

% \textbf{Palabras clave}: Prácticum II, Formación Profesional, domótica, ciberseguridad, enseñanza práctica, aprendizaje basado en problemas.



% Durante la observación en el aula, pude identificar dinámicas interesantes en la interacción profesor-alumno, así como detectar dificultades individuales. Entre los casos destacados:
% \begin{itemize}
%     \item Un alumno de \textit{1º GM} mostraba signos de falta de concentración y comentó que no podía dormir bien. Tras una breve conversación en la que le sugerí buscar ayuda profesional, retomó la práctica con mayor enfoque.
%     \item Otro estudiante de \textit{1º GM} presentó una descripción ambigua en una práctica. Tras hacerle reflexionar sobre la claridad de su explicación desde el punto de vista de un cliente, modificó su respuesta.
%     \item Un alumno me llamó \textit{"profe"} y me preguntó sobre disyuntores. Tras explicarle que no podía ayudarle porque no era mi área, entablamos una conversación en la que cuestionó por qué me habían asignado a un módulo eléctrico, reflejando la falta de alineación entre mi perfil y la asignación de prácticas.
% \end{itemize}

% La evaluación de la actividad evidenció que la parte práctica generó gran interés en el alumnado, destacando la importancia de incluir más contenido de ciberseguridad en la formación de Instalaciones Domóticas. A pesar de las limitaciones derivadas de mi perfil técnico, esta experiencia me permitió reflexionar sobre el papel del docente en FP, el enfoque metodológico en enseñanza práctica y la importancia de la motivación y adaptación de los contenidos a las necesidades del alumnado.


